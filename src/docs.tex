\documentclass[12pt]{article}
\usepackage{amsmath}
\usepackage{graphicx}
\usepackage{hyperref}
\usepackage[latin1]{inputenc}

\title{Multilayered perceptron}
\author{bgreydon, pintoved, cgreenbe}
\date{22/11/2022}

\begin{document}
\maketitle

\newpage
\section{brief introduction}
The graphical interface of the 3D viewer is written using the qt graphical library.
Model perceptron was developed and tested on the Windows and Linux with kde and gnome graphical shell of the desktop environment.

The program provide the ability to form and train neural network models to classify handwritten Latin letters.

The perceptron can:
\begin{itemize}
  \item classify images with handwritten letters of the Latin alphabet;
  \item have from 2 to 5 hidden layers;
  \item use a sigmoid activation function for each hidden layer;
  \item learn on an open dataset EMNIST;
  \item be trained by backpropagation.
\end{itemize}

The program can:
\begin{itemize}
  \item run tests on test samples followed by output average accuracy, precision, recall, f-measure ant total spent time;
  \item load bmp image with latin letters;
  \item work with the drawn image in real time;
  \item real time training and cross-validation training;
  \item swith the number of hidden layers;
  \item save and load weights of the percetron.
\end{itemize}

\end{document}
